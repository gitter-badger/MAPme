\documentclass[12pt,a4paper,oneside]{report}

% Formatting stuff:
\setlength{\textheight}{22cm}
\setlength{\textwidth}{16cm}
\setlength{\oddsidemargin}{0cm}
\setlength{\evensidemargin}{0cm}
\setlength{\topmargin}{0cm}
\setlength{\parindent}{0cm}
\setlength{\parskip}{0.8ex}

% Packages required
\usepackage{listings}
\usepackage{natbib}
\usepackage{fancyhdr}
\usepackage[english]{babel}
\usepackage[utf8]{inputenc}
\usepackage[normalem]{ulem}
\usepackage{setspace}
\usepackage{url}
\usepackage{graphicx}
\usepackage{float}
\usepackage{titlesec}
\usepackage{appendix}
\useunder{\uline}{\ul}{}
\def\code#1{\texttt{#1}}
\setcounter{secnumdepth}{4}
\renewcommand\thesection{\arabic{section}}

% Start
\begin{document}

% Title page
\begin{titlepage}
	\centering
	{\scshape\LARGE User Interface Report \par}
	\vspace{1cm}
	{\scshape\Large MAPme Android Application\par}
	\vspace{1.5cm}
	{\LARGE Authors: D.G. Smith, P.S. Sebeikin\par}
	\vspace{2cm}
	{\large \today\par}
\end{titlepage}
\tableofcontents
\pagebreak
% Body
% discusses the problem area that we are dealing with

\section{Target Audience}
The target audience remains the elderly bracket of people; predominantly retirees.  The target audience is assumed to have a basic knowledge of mobile apps and how they function.  The target audience is also assumed to have a basic knowledge of how to operate a smart phone photo camera.

\section{Project Context}
The assumption is that users of the application are pre-registered on the Animal Demography Unit (ADU) website, however they can be redirected to the website via a link in the MAPme app.  The app will be used predominantly in the field, possibly in areas where wireless connectivity is limited or non-existent.  Therefore, it is imperative that the MAPme app is able to locally store records and images until such a time as wireless connectivity becomes available.

\section{Functional Requirements}
	\subsection{Mandatory}
		\begin{itemize}
			\item A registration screen that must require the user's email, password and ADU number.
			\item User profile can be viewed and default settings used for the templating feature can be changed. The user's name, email and ADU cannot be changed from within the app. This can only be achieved by navigating to the Virtual Museam website.
			\item Photos must be taken from the camera (independent of the app) or the camera must be accessed from within the app.
			\item Users should be able to browse existing photos from the device's gallery.
			\item Users should be able to review the photos before submission, possibly removing unwanted selections.
			\item Use of global positioning system (GPS) coordinates:
				\begin{itemize}
					\item Interfaces with phone GPS location service.
					\item Entered manually by user.
					\item Google Maps widget
				\end{itemize}
			\item Photos should be uploaded immediately or later. This alludes to the probability of a mobile phone based temporary data storage structure that holds recorded data until such a time as its upload to the ADU server is initiated.
			\item Users should have the ability to save a template of the most recent new record submission.  Users should be prompted about using the previously saved template or starting a new submission with a blank template.
			\item In the case of animal captures, sound files should be recorded and uploaded (app should interface with the device's voice recorder).
			\item User should be able to enter a locality description.
			\item Date and time is automatically recorded from the device or metadata of the image. The user should be able to update this as necessary.
			\item The application should enforce mandatory entries and allow optional entries.
			\item The user must be able to submit to a particular project (e.g. OrchidMAP, FrogMAP, MammalMAP etc.).
			\item Include a history of user record submissions, to be stored locally on the device that is being used for the submissions.
		\end{itemize}

	\subsection{Secondary}
		\begin{itemize}
			\item The user should be able to:
			 	\begin{itemize}
					\item specify the species.
					\item choose an environmental description.
					\item indicate whether a flower and/or fruit is present.
					\item choose the number of specimens observed.
					\item choose if the specimen is natural or cultivated.
					\item choose the growth form of the specimen.
					\item choose language preferences (defaults to english).
			 	\end{itemize}
	  \end{itemize}

\section{Client Feedback}
	On 13 May 2016, Dylan Smith and Paul Sebeikin, hence forth known as the developers, conducted a meeting with Dr. Craig Peter, hence forth known as the client.  The following feedback was received:

\begin{itemize}
	\item Regardless of whether the date of the image is saved along with the image, the date must appear on the ``New Record" screen to give the user piece of mind that the date has been captured correctly.
	\item The date should be automatically collected from the image metadata and presented to the user and the user should have the ability to change this, though the client has stipulated that in all but 5\% of cases, a change of date is unlikely.
	\item The client indicated that in addition to the mandatory ``New Record" fields that the developers, the ``country", ``province" and ``nearest town" fields were also mandatory.
	\item The client suggested incorporating the date field into the initial ``New Record" screen and moving the ``locality description" field to the second screen along with the newly mandatory ``country", ``province" and ``nearest town" fields.
	\item A discussion was had about geo-tags in images:
	\begin{itemize}
		\item The developers communicated to the client that geo-tagging of photos is turned off by default on most smart phones.  It was deduced that it would be difficult to enforce geo-tagging on images if this feature is turned off and the user wishes to quickly take a photograph without interfacing with the MAPme app directly.
		\item  A suggestion was put forward for the app to incorporate a minimal Google Maps widget which can be used to pinpoint a geographic location in the event that an image is taken without a geo-tag, outside of the MAPme app.
		\item The client approved of this suggestion.
	\end{itemize}
	\item The client noted the importance of a template system for new record insertion.  Users would prefer to enter as little data as possible in the shortest amount of time therefore, if various default values are inserted into various fields of the ``New Record" screen or if previously submitted record data is preserved for consequent use, the process would be far quicker to execute.
	\item The client demonstrated the use of an app called AndroSense which appeared to calculate alititude based on GPS coordinates of the device.
	\end{itemize}

\section{Changes}
	Based on feedback acquired from the client, the developers will execute the following changes to the user interface mockup from the initial prototype demonstration:
\begin{itemize}
	\item A profile screen will be created for the MAPme app in which default settings can be tracked and changed by the user.  These settings will be used to supply default values to new record submissions in an effort to reduce the record submission times and improve general application efficiency.
	\item The homescreen was adpated to incorporate a link to the  ``Profile" screen at the expense of the ``About" link. ``About" is now solely accesible from the menu of the MAPme action bar.
	\item Upon the submission of a new record, the user will have the option to save the data they have submitted to a single template.  When an attempt is made to insert a further record, the user will have the option to make use of the previously saved template or create a new record submission with default values.
	\item Three new mandatory fields were added namely Country, Province and Nearest Town to the ``New Record" screen.
	\item The Date field is now displayed with the option to update. As a result, the Description field was moved to the next screen.
	\item Updated alignment of textboxes and labels to comply with human computer interaction (HCI) standards.
\end{itemize}

\section{Workflows and Use Case Diagram}
Below are the workflows
\end{document}
